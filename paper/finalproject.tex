%-----------------------------------------------------------------------------
%               CSCI5525 Final Project LaTeX file.
%-----------------------------------------------------------------------------

\documentclass[preprint]{sigplanconf}

% The following \documentclass options may be useful:
%
% 10pt          To set in 10-point type instead of 9-point.
% 11pt          To set in 11-point type instead of 9-point.
% authoryear    To obtain author/year citation style instead of numeric.

\usepackage{graphicx}
\usepackage{listings}
\usepackage{verbatim}
\usepackage{amsmath}
\usepackage[pdfborder={0 0 0}]{hyperref}

\lstset{
basicstyle=\small,
numberstyle=\small,
commentstyle=\textit
stepnumber=1
}

\begin{document}

\conferenceinfo{CSCI5525 '11}{November 16th, 2011, Boulder,CO.} 
\copyrightyear{2011} 
\copyrightdata{[Like a Boss.]} 

\titlebanner{Reference Counting in P3}        % These are ignored unless
\preprintfooter{An implementation of reference counting for P3}   % 'preprint' option specified.

\title{Reference Counting in P3}
%\subtitle{Subtitle Text, if any}

\authorinfo{Rob Elsner}
           {University of Colorado, Boulder}
           {rob.elsner@colorado.edu}
\authorinfo{Brent Smith}
           {University of Colorado, Boulder}
           {brent.m.smith@colorado.edu}

\maketitle

\begin{abstract}
Automatic memory management is a common feature in modern languages, eliminating the need to
manually allocate and deallocate memory at the proper program points.  This feature helps to
eliminate costly errors that may go undetected in short lived programs, but cause long running 
programs to eventually fail. 
In this paper, we will discuss the implementation of a reference counting
garbage collector for a subset of the Python programming language.
\end{abstract}

\category{CR-number}{subcategory}{third-level}

\terms
memory management, garbage collector

\keywords
keyword1, keyword2

\section{Introduction}

There are currently two widely used strategies to perform automatic memory management.  We will
discuss both strategies briefly, consider the tradeoffs associated with each, and then discuss
how the reference counting scheme fits in to the current architecture of our compiler.

Tracing garbage collection is ....
%ref{mccarthy}
Tracing garbage collection is commonly associated with the garbage collector implementation  
provided by the Java programming language, but many languages have libraries that provide GC 
implementations 
%\ref{c++ implementations}
or allow the user the option of manual memory management or a tracing GC.
%\ref{Apple's objective C framework}

Reference counting \cite{Collins:1960} is way cooler since you don't suffer long pause times, but 
may not achieve equivalent throughput to tracing garbage collection.  It's also wicked hard 
to get right on multi-threaded contexts.  You have to be an evil genious to get this shit right,
and use atomic compare and swap operations, whatever those are.

\begin{figure}
\normalsize % use normal font size in the figure
\input{slot_runtime.tex}
\caption{Benchmark results on Local Cluster}
\label{fig:slot_runtime}
\end{figure}


\appendix
\section{Appendix of Appendices}

Here we list the categories of topics discussing different types of thingies 
and such as, whatnot and therefore. Thank you.

\acks

I'd like to thank all the beautiful children.

% We recommend abbrvnat bibliography style.
\bibliographystyle{abbrvnat}
% Use bibtex instead of the other crap
\bibliography{finalproject}

% The bibliography should be embedded for final submission.
%\begin{thebibliography}{}
%\softraggedright
%\bibitem[Smith et~al.(2009)Smith, Jones]{smith02}
%P. Q. Smith, and X. Y. Jones. ...reference text...
%\end{thebibliography}

\end{document}
